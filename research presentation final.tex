\documentclass{beamer}
\usetheme{Boadilla}

\title{Challenging Genetic Algorithms for Artificial Intelligence with Random Multi-opponent Games}
\author{Isa bin Mohd Faizal, Muhammad Amir Rafiq}
\institute{Universiti Kebangsaan Malaysia}

\section{Introduction}
\section{Problem Statement}
\section{Objective}
\section{Methodology}
\section{Results}
\section{Discussion}
\section{Conclusions}

\begin{document}
	
\begin{frame}[plain]
    \maketitle
\end{frame}

\begin{frame}{Outline}

	\tableofcontents
	
\end{frame}

\begin{frame}{Introduction}
	
	\begin{itemize}
		\item  Artificial intelligence can beat even the best human players in various video games.
		\item Stockfish and AlphaGo
		\item However, most of these games are fully deterministic.
		\item There is only one possible starting point, which sets a base for neural networks and AI to iterate upon.
	\end{itemize}
	What would happen if an AI needed to learn from starting a random gamestate? Will this affect its performance?
	
\end{frame}

\begin{frame}{Problem Statement}

	\begin{itemize}
		
		\item How well will a genetic model perform in multi-opponent games of random chance?
		\item How long will it take until it reaches a performance plateau?
		\item How fast will it improve over time during training?
		
		
	\end{itemize}

\end{frame}

\begin{frame}{Objective}
	
	\begin{itemize}
		
		\item Improve on past models of artificial intelligence using elements of randomness
		\item Incorporating elements of randomness into the model itself to prevent premature plateauing of performance
		\item Games of luck are particularly prone to problems of plateauing performance due to lower correlation between skill and winning

	\end{itemize}
	
\end{frame}

\begin{frame}{Methodology}
	
	\begin{itemize}
	
		\item Sets of 100 neural networks are generated randomly
		\item Each neural network will play against other models in its generation.
		\item Performance of the neural network is measured by a point system based on how many cards the NN is holding at the end of the round.

	\end{itemize}
	
\end{frame}

\begin{frame}{Methodology}
	
	\begin{itemize}
		
		\item The models will be sorted according to their performance by amount of points.
		\item The \verb|eaSimple| algorithm, provided by the \texttt{DEAP} python package, will handle crossover and mutation to generate a new population.
		\item The crossover probabily was set to 0.8 (80\%), and the mutation probabily was set to 0.01 (1\%).
		\item The mean, lowest and highest score will be tracked over time.
		
	\end{itemize}
	
\end{frame}

\begin{frame}{\texttt{eaSimple - DEAP}}

\end{frame}

\begin{frame}{Results}
	
	\begin{itemize}
		\item We tested 1000 generations of neural networks playing Big2, taking about 20 hours of computer time on a home computer.
		\item The results are as follows according to these graphs:
	\end{itemize}
	
\end{frame}

\begin{frame}
	content...
\end{frame}

\begin{frame}{Discussion}
	
\end{frame}

\begin{frame}{Conclusion}

\end{frame}


\end{document}
