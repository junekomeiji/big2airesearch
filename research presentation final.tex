\documentclass{beamer}
\usetheme{Boadilla}
\usepackage{pgfplots}

\title{Challenging Genetic Algorithms for Artificial Intelligence with Random Multi-opponent Games}
\author{Isa bin Mohd Faizal, Muhammad Amir Rafiq}
\institute{Universiti Kebangsaan Malaysia}

\section{Introduction}
\section{Problem Statement}
\section{Objective}
\section{Methodology}
\section{Results}
\section{Discussion}
\section{Conclusions}

\begin{document}
	
\pgfplotstableread{gen3.dat}{\table}
	
\begin{frame}[plain]
    \maketitle
\end{frame}

\begin{frame}{Outline}

	\tableofcontents
	
\end{frame}

\begin{frame}{Introduction}
	
	\begin{itemize}
		\item  Artificial intelligence can beat even the best human players in various video games.
		\item Stockfish and AlphaGo
		\item However, most of these games are fully deterministic.
		
	\end{itemize}
	
	
\end{frame}

\begin{frame}{Problem Statement}

	\begin{itemize}
		
		\item How well will a genetic model perform in multi-opponent games of random chance?
		\item How long will it take until it reaches a performance plateau?
		\item How fast will it improve over time during training?
		
		
	\end{itemize}

\end{frame}

\begin{frame}{Objective}
	
	\begin{itemize}
		
		\item Improve on past models of artificial intelligence using elements of randomness
		\item Incorporating elements of randomness into the model itself to prevent premature plateauing of performance
		\item Games of luck are particularly prone to problems of plateauing performance due to lower correlation between skill and winning
		
	\end{itemize}
	
\end{frame}

\begin{frame}{Methodology}
	
	\begin{itemize}
	
		\item Sets of 100 neural networks are generated randomly
		\item Each neural network will play against all other models
		\item Performance of the neural network is measured by a point system based on the placement of the neural networks
		
	\end{itemize}
	
\end{frame}

\begin{frame}{Methodology}
	
	\begin{itemize}
		
		\item The models will be sorted according to their performance by amount of points
		\item The top 50th percentile of neural networks will be allowed to reproduce with slight mutation to create two successors
		\item The mean, lowest and highest score will be tracked over time.
		
	\end{itemize}
	
\end{frame}

\begin{frame}{Results}
	
	\begin{itemize}
		\item We tested 1000 generations of neural networks playing Big2, taking about 20 hours of computer time on a home computer.
		\item The results are as follows according to these graphs:
	\end{itemize}
	
\end{frame}

\begin{frame}{Results}
	\begin{tikzpicture}
		\begin{axis}[
			xmin = 0, xmax = 1000,
			ymin = -15, ymax = 0,
			xtick distance= 100,
			ytick distance= 1,
			major grid style={lightgray},
			minor grid style={lightgray!25},
			width= \textwidth,
			height = 0.75\textwidth,
			legend cell align = {left},
			legend pos = north west	
		]
		
		\addplot table[x=Gen, y=Min] {\table};
		\end{axis}
	\end{tikzpicture}
\end{frame}

\begin{frame}{Results}
	\begin{tikzpicture}
		\begin{axis}[
			xmin = 0, xmax = 1000,
			ymin = -15, ymax = 0,
			xtick distance= 100,
			ytick distance= 1,
			major grid style={lightgray},
			minor grid style={lightgray!25},
			width= \textwidth,
			height = 0.75\textwidth,
			legend cell align = {left},
			legend pos = north west	
			]
			
			\addplot table[x=Gen, y=Mean] {\table};
		\end{axis}
	\end{tikzpicture}
\end{frame}

\begin{frame}{Results}
	\begin{tikzpicture}
		\begin{axis}[
			xmin = 0, xmax = 1000,
			ymin = -13, ymax = 30,
			xtick distance= 100,
			ytick distance= 5,
			major grid style={lightgray},
			minor grid style={lightgray!25},
			width= \textwidth,
			height = 0.75\textwidth,
			legend cell align = {left},
			legend pos = north west	
			]
			
			\addplot table[x=Gen, y=Max] {\table};
		\end{axis}
	\end{tikzpicture}
\end{frame}

\begin{frame}{Discussion}
	
\end{frame}

\begin{frame}{Conclusion}

\end{frame}


\end{document}
